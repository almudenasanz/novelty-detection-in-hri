% Appendix C



\chapter{Regulatory Framework} % Main appendix title

\label{AppendixB} % For referencing this appendix elsewhere, use \ref{AppendixA}

\lhead{Appendix B. \emph{Regulatory Framework}} % This is for the header on each page - perhaps a shortened title

\textbf{The software developed by the author} for this Thesis is available in \cite{github}, including the main system and the experiments performed. The code is licensed under the GPLv3, which states that every user has "the freedom to use the software for any purpose; the freedom to change the software to suit your needs; the freedom to share the software with your friends and neighbors; and
the freedom to share the changes you make" \cite{gnu}. Any software using or importing GPLv3 licensed code is obligued to be GPLv3 too \cite{gnu-obli}. It must also disclose the source, state any changes made to the original code and include the original code \cite{gnu-obli}.

The version of Python \cite{open_source} used in the software is 2.7. It is declared Open Source and GPL compatible. "Open source software is software that can be freely used, changed, and shared (in modified or unmodified form) by anyone." \cite{open_source}

The \textbf{ScikitLearn Library} \cite{scikit-learn} is where some of the novelty detection algorithms used in the experiments of this Thesis are implemented. It holds a simplified or new BSD license \cite{bsd}, stating that their libraries are open source and commercially usable.

One of the novelty deetction algorithms, \textbf{Least Squares Anomaly Detection} (LSA), was developed by Jonh Quinn \cite{lsa}. The python implementation file can be found in his webpage. After contacting him, he denominated his implementation as License-free. 

The \textbf{IPython} 2.0 toolset \cite{ipython}, was used to generate the notebooks where the system code is displayed. It also holds a new BSD license. Which they state in their webpage \cite{ipy_l}. 

Other Python libraries used in the software are:

\textbf{Numpy} \cite{numpy}: simplified or new-license BSD \cite{bsd}. They publish it explicitly in their webpage \cite{numpy_l}
\textbf{Pandas} \cite{pandas}: simplified or new-license BSD \cite{bsd}. Can be found at \cite{pandas_l}
\textbf{Matplotlib} \cite{matplotlib}:  only uses BSD compatible code. License can be found at \cite{matplotlib_l}

GPLv3 can be deniminated \emph{viral}, in the sense that any code importing a GPLv3 licensed software must be GPL v3 too. Also, modifiyng GPLv3 licensed code, means that you are obligued to release it. With this in mind, any new application that uses the sofware designed in this Thesis, should be made public. As the libraries and tools used for this Thesis have all a BSD license or are unliensed, there was no obligation to release the code, since it is not required by the license. However, the decision of choosing a GPL v3 was made thoughtfully, because I, as the author, understand that Open Source software benefit future academic research. Since I could not have done any of these work without the Open Source tools I have used, I think that it is only fair publishing the code for the application so others can reuse it. In my opinion, research can only move forward if we share our applications and discoveries with others and keep "standing on the shoulders of giants"\footnote{Phrase known as an expression of Isaac Newton in a letter to Robert Hook in 1676}.

\begin{flushright}

\end{flushright}