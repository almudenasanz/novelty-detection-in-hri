% Chapter 7

\chapter{Epilogue. Socioeconomic State of Artificial Intelligence} % Main chapter title

\label{Epilogue} % For referencing the chapter elsewhere, use \ref{Chapter1} 

\lhead{Epilogue. \emph{Socioeconomic State of Artificial Intelligence}} % This is for the header on each page - perhaps a shortened title

Artificial Intelligence (A.I.) is becoming more and more adavanced in these past years. Self driving cars are already on the streets \cite{google}, and new A.I. tools are even occupying seats in the board of directors of some companies \cite{board}. In their release, a senior partner of the mentioned firm, Deep Knowledge Ventures, saids that they “We were attracted to a software tool that could in large part automate due diligence and use historical data-sets to uncover trends that are not immediately obvious to humans that are surveying top-line data.” \cite{board}. The access to huge amounts of data and their fast analysis can provide answers that outsmart any other member of the board of directors.

\citeauthor{hawking} has recently written an article on the issue of Artificial Intelligence where he states that the potential benefits are enormous \cite{hawking}. If we, as human intellicence, have been able to create everything that civilization has to offer, the magnification of this intelligence by the powerful tools of A.I. is unpredictible. He mentions that the eradication of war, disease, and poverty would be high on anyone's list with the help of A.I. Sucess in developing and increasing the applications and performace of AI "would be the biggest event in human history" \cite{hawking}. 

However, the development A.I. comes along with great risks. "Imagine such technology outsmarting financial markets, out-inventing human researchers, out-manipulating human leaders, and developing weapons we cannot even understand" \citeauthor{hawking} says. "Whereas the short-term impact of A.I. depends on who controls it, the long-term impact depends on whether it can be controlled at all" \cite{hawking}. According to Hawking, "we are facing potentially the best or worst thing to happen to humanity in history" \cite{hawking}.

One may think that high tech companies with great resources have access to more sophisticated A.I, and that they are able to obtain more applications and benefits. But A.I. is also being developed by Open Source libraries, such as SciKitLearn \cite{scikit-learn}, which give free and unlimited usage of this tools to anyone with internet access and basic notions of programming. The applications that can be achieved with these libraries have also a huge potential, in fact, they made this Thesis possible. However, putting them in the hands of anyone who wants to use them also attains other risks in the short term, as anything that is of public use.

There is not an official organism dedicated to study legal regulations for A.I. There is a recently created field called Roboethics that deals with this issue, stated as "how humans design, construct, use and treat robots and other artificially intelligent beings" \cite{wiki-roboethics}. \citeauthor{roboethics} with the collabortion of EURON \cite{EURON}, proposed a Roboethics Roadmap \cite{roadmap} in 2006, where they stated the formal definitions involved in the subject, and proposed a roadmap to follow. In 2008, he wrote an article on the topic as well \cite{roboethics}, where he explained the main issues surrounding Roboethics. Since then, the international community has organizad conferences on the topic, and initiatives such as the Open Roboethics initiative (ORi) \cite{ori}, that aims to foster discussions in roboethics by means of mass colaboration, have emerged.

Since A.I. has so many implications in our future, we should make an effort to ensure the best outcome. I think it is really important to invest resources and time in the field of Roboethics in the short term, and come to a consensus of how approach A.I. and its applications.   
 

\begin{flushright}

\end{flushright}
